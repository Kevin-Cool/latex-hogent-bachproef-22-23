%==============================================================================
% Sjabloon onderzoeksvoorstel bachproef
%==============================================================================
% Gebaseerd op document class `hogent-article'
% zie <https://github.com/HoGentTIN/latex-hogent-article>

% Voor een voorstel in het Engels: voeg de documentclass-optie [english] toe.
% Let op: kan enkel na toestemming van de bachelorproefcoördinator!
\documentclass{hogent-article}

% Invoegen bibliografiebestand
\addbibresource{voorstel.bib}

% Informatie over de opleiding, het vak en soort opdracht
\studyprogramme{Professionele bachelor toegepaste informatica}
\course{Bachelorproef}
\assignmenttype{Onderzoeksvoorstel}
% Voor een voorstel in het Engels, haal de volgende 3 regels uit commentaar
% \studyprogramme{Bachelor of applied information technology}
% \course{Bachelor thesis}
% \assignmenttype{Research proposal}

\academicyear{2022-2023}

% TODO: Werktitel
\title{Open source tool voor het verhogen van effiecientie tijdens studeren doormidel van dynamies aangemaakte vragen gebaseerd op de gebruiker zijn retentie van informatie }

\author{Cool Kevin}
\email{kevin.cool@student.hogent.be }


% TODO: Geef de co-promotor op
\supervisor[Co-promotor]{S. Beekman (Synalco, \href{mailto:sigrid.beekman@synalco.be}{sigrid.beekman@synalco.be})}

% Binnen welke specialisatierichting uit 3TI situeert dit onderzoek zich?
% Kies uit deze lijst:
%
% - Mobile \& Enterprise development
% - AI \& Data Engineering
% - Functional \& Business Analysis
% - System \& Network Administrator
% - Mainframe Expert
% - Als het onderzoek niet past binnen een van deze domeinen specifieer je deze
%   zelf
%
\specialisation{AI \& Data Engineering}
\keywords{studie efficientie, dynamiese vragen aanmaken, studie partoon }

\begin{document}

\begin{abstract}
 Studeren is een proces dat veel tijd en energie inneemt, een proces dat iedereen anders benadert. De bedoeling van deze bachelorproef is om een tool te ontwikkelen dat het proces van studeren meer persoonlijk en efficiënter zal maken. De tool zal door middel van NLP de informatie in een gegeven tekst kunnen opnemen, en zal dan dynamisch vragen aanmaken gebaseerd op het leerpatroon van de gebruiker. Doordat de vragen dynamisch aangemaakt worden kan er ook voor variatie gezorgd worden wat zal tegenwerken dat de gebruiker enkel de antwoorden uit het hooft zal leren. Er wordt verwacht dat door gebruik van de tool een gebruiker niet alleen simpeler zal kunnen studeren maar dat er ook een beter begrip zal zijn van de informatie.

\end{abstract}

\tableofcontents

% De hoofdtekst van het voorstel zit in een apart bestand, zodat het makkelijk
% kan opgenomen worden in de bijlagen van de bachelorproef zelf.
\input{voorstel-inhoud}

\printbibliography[heading=bibintoc]

\end{document}